\documentclass[12pt,a4paper]{article}
\usepackage[latin2]{inputenc}
\usepackage{graphicx}
\usepackage{ulem}
\usepackage{amsmath}
\usepackage{multicol}
\begin{document}
\begin{center}DAA432C \end{center}
\begin{center}Group-16 Assignment-01\end{center}

\begin{center}\textit{B. Tech IT 4$^{th}$ Semester Sec-B}
\end{center}

\begin{center}\textit{Indian Institute of Information Technology, 
Allahabad}\end{center}


\begin{multicols}{3}
\begin{center}IIT2019099\end{center}

\begin{center}Nitesh Rawat\end{center}

\begin{center}iit2019099@iiita.ac.in\end{center}

\begin{center}IIT2019100\end{center}

\begin{center}Maitry Jadiya\end{center}

\begin{center}iit2019100@iiita.ac.in\end{center}

\begin{center}IIT2019101\end{center}

\begin{center}Aryan Gupta\end{center}

\begin{center}iit2019101@iiita.ac.in\end{center}
\end{multicols}



\begin{multicols}{2}
\textbf{\textit{Abstract--- }In this report, we've shown the design and analysis of an algorithm that finds the missing element in an array that represents elements of an arithmetic progression using divide and conquer algorithm.}

\textbf{\textit{Keywords--- } Arithmetic progression, Divide and conquer, array, Binary Search }

\begin{center}I. INTRODUCTION\end{center}

This paper discusses about an algorithm that is  designed to  find the missing element in an array that represents elements of an arithmetic progression in order using divide and conquer approach.

\ \ \ \  An arithmetic sequence or progression(AP) is defined as a sequence of numbers in which for every pair of consecutive terms, the second number is obtained by adding a fixed number to the first one.

\ \ \ \ The array is already sorted either in increasing or decreasing order because, the elements of an array represent an arithmetic progression . 
 
\ \ \ \ This paper also contains the analysis about the time and space complexities of the algorithm. By the end of the paper, we will be able to understand all the components of algorithm design and will learn different ways of analysing the 
algorithms. 





\begin{center}VII. CONCLUSION\end{center}

\ \ \ \ We can conclude that the above algorithm has the least time and space 
complexity to find the missing element in an array that represents elements of an arithmetic progression in order.


\begin{center}REFERENCES\end{center}

$[$1$]$ Introduction to Algorithms / Thomas H. Cormen \ldots $[$et 
al.$]$. - 3$^{rd}$ edition.

$[$2$]$ The Design and Analysis of Algorithms (Pearson) by A V Aho, J E 
Hopcroft, and J D Ullman 

$[$3$]$ Algorithm Design (Pearson) by J Kleinberg, and E Tard

$[$4$]$ https://www.geeksforgeeks.org/find-missing-number-arithmetic-progression/

\end{multicols}







\end{document}
