\documentclass[12pt,a4paper]{article}
\usepackage[latin2]{inputenc}
\usepackage{graphicx}
\usepackage{ulem}
\usepackage{amsmath}
\usepackage{multicol}
\begin{document}
\begin{center}DAA432C \end{center}
\begin{center}Group-16 Assignment-01\end{center}

\begin{center}\textit{B. Tech IT 4$^{th}$ Semester Sec-B}
\end{center}

\begin{center}\textit{Indian Institute of Information Technology, 
Allahabad}\end{center}


\begin{multicols}{3}
\begin{center}IIT2019099\end{center}

\begin{center}Nitesh Rawat\end{center}

\begin{center}iit2019099@iiita.ac.in\end{center}

\begin{center}IIT2019100\end{center}

\begin{center}Maitry Jadiya\end{center}

\begin{center}iit2019100@iiita.ac.in\end{center}

\begin{center}IIT2019101\end{center}

\begin{center}Aryan Gupta\end{center}

\begin{center}iit2019101@iiita.ac.in\end{center}
\end{multicols}



\begin{multicols}{2}
\textbf{\textit{Abstract--- }In this report, we've shown the design and analysis of an algorithm that finds the missing element in an array that represents elements of an arithmetic progression using divide and conquer algorithm.}

\textbf{\textit{Keywords--- } Arithmetic progression, Divide and conquer, array, Binary Search }

\begin{center}I. INTRODUCTION\end{center}

This paper discusses about an algorithm that is  designed to  find the missing element in an array that represents elements of an arithmetic progression in order using divide and conquer approach.

\ \ \ \  An arithmetic sequence or progression(AP) is defined as a sequence of numbers in which for every pair of consecutive terms, the second number is obtained by adding a fixed number to the first one.

\ \ \ \ The array is already sorted either in increasing or decreasing order because, the elements of an array represent an arithmetic progression . 
 
\ \ \ \ This paper also contains the analysis about the time and space complexities of the algorithm. By the end of the paper, we will be able to understand all the components of algorithm design and will learn different ways of analysing the 
algorithms. 

\begin{center}II. ALGORITHM DESIGN\end{center}

\ \ \ \ The given problem can be solved by divide and conquer algorithm.Here we will be using the binary search approach .We will divide a given problem into smaller sub-problems and appropriately combine their solutions to get the solution to the main problem.  


\textit{  Approach:}Idea is to compare the elements of the given array(A) with an array whose elements are in proper arithmetic progression(B). We will find the first index of mismatch of A and B.The element of B in this index will give us our missing element.


\textit{  Algorithm:}

\newcounter{numberedCntA}
\begin{enumerate}
\item Find the mid element of the array every time search range is divided and initialise a result variable which will keep index of mismatched index.Input array A with a missing element and an array B which has elements in proper arithmetic progression(also having the missing element of A) is taken.
\item Check the values of array A and B on the mid index and if the elements are same then it would mean that no element was missing in the AP till this index.In this case start searching only to the right of mid(right half of the current search range).
\item Check the values of array A and B on the mid index and if the elements are unequal then store the index in result and keep checking to the left of mid(left half of the current search range) as the minimum index of mismatched value is required.If a smaller index of mismatch is found then result is updated with this index.
\item The value of result is returned after the range becomes zero.
\item After performing all the steps for all the subproblems, if the value of result variable is unchanged, then no element is missing in the array, otherwise, print the value of the result variable.
\setcounter{numberedCntA}{\theenumi}
\end{enumerate}




\begin{center}III. PSEUDO CODE\end{center}

\textit{ Declare global variable ans=INT\_MIN}

Function binarysearch(Argument a$[$$]$,Argument b$[$$]$, Argument n) 


\{ 
\quad initialize l =0 , h=n-1 ;
\quad  while l is less than h 
\{
\quad initialize m = l+((h-l) / 2);

\quad initialize res =-1;



\quad If a$[m]$==b$[m]$ l=m+1 range squeezes to right half


\quad Else if a$[m]$!=b$[m]$  res=mid, end=mid-1;


\}
 

\quad return res; 

end

\} 

\ \ Main function()\{ 

\quad Initialize integer array arr$[$$]$

\quad  Initialize n as size of array

\quad Input the elements of the array 

\quad d=a[n-1]-a[0]/n;
\quad b[0]=a[0] ; 
\quad for 1 to n b[i]=b[i-1]+d

\quad ans=binarysearch(a, b, n)

\quad print ans

\}



\} 



\begin{center}VII. CONCLUSION\end{center}

\ \ \ \ We can conclude that the above algorithm has the least time and space 
complexity to find the missing element in an array that represents elements of an arithmetic progression in order.


\begin{center}REFERENCES\end{center}

$[$1$]$ Introduction to Algorithms / Thomas H. Cormen \ldots $[$et 
al.$]$. - 3$^{rd}$ edition.

$[$2$]$ The Design and Analysis of Algorithms (Pearson) by A V Aho, J E 
Hopcroft, and J D Ullman 

$[$3$]$ Algorithm Design (Pearson) by J Kleinberg, and E Tard

$[$4$]$ https://www.geeksforgeeks.org/find-missing-number-arithmetic-progression/

\end{multicols}







\end{document}
